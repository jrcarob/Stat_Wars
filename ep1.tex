\documentclass{beamer}
\usepackage[spanish]{babel}
\usetheme{StarWars}   

\title[Episode V]{Star Wars}
\subtitle{Episodio I : Introducci\'on a la Estad\'istica y Probabilidad}
\author{Obi Wan Kenobi} 

\begin{document}   
    
  \pagetitle 
  
  
\begin{frame}
\frametitle{\'Indice}
\tableofcontents
\end{frame}
 
  
\section{Relaci\'on de problemas}          
\begin{frame}
\frametitle{Problema 1\ \small{\textit{\color{red}{(tomado de Tito Eliatron @eliatron)}}}}
El recuento de midiclorianos en un humano medio sigue una distribuci\'on normal de media 3500 unidades por c\'elula y desviaci\'on t\'ipica 500. Un ser humano es considerado Jedi si su recuento de midiclorianos por c\'elula sobrepasa los 5000.

\begin{itemize}
\item[(a)] ?`Cu\'al es la probabilidad de que un ser humano sea Jedi?
\item[(b)] En Tattooine hay 1500 seres humanos trabajando en las Granjas de Humedad. ?`Cu\'antos Jedis esperamos encontrar entre ellos?
\item[(c)] ?`Cu\'al es la probabilidad de que haya 5 o m\'as Jedis entre los granjeros de humedad?
\item[(d)] Sabiendo que una persona no es Jedi, ?`cu\'al es la probabilidad de que tenga m\'as de 4000 midiclorianos?
\end{itemize}
\texttt{\color{red}{https://www.youtube.com/watch?v=DML07xIy-ro\&t=7s}}
\end{frame}
  
%\section{Soluciones} 
\begin{frame}
\frametitle{Problema 2 \small{\textit{\color{red}{(adaptado de Virgilio G\'omez-Rubio @precariobecario)}}}}
La potencia de fuego de la \textit{Estrella de la Muerte} es una temible arma de guerra al servicio del Imperio. Los ingenieros de la flota estelar han estudiado la distancia (en \textit{unidades astron\'omicas\footnote{una unidad astron\'omica (ua) equivale a la distancia media Sol-Tierra y mide 149.597.870 km. Se usa para distancias dentro de los sistemas planetarios.}}) a la que puede llegar el disparo y han observado que sigue la siguiente densidad de probabilidad:

\vspace{-0.5cm}
\begin{equation*}
f_{X}(x)=\left\{\begin{array}{ll}
k\cdot x & 0 \leq x< 30 \\
0 & \text{en\ otro\ caso}
\end{array}\right.
\end{equation*}

\begin{itemize}
\item[(a)]Calcular el valor de \textit{k} para que $f_{X}(x)$ sea una funci\'on de densidad de probabilidad v\'alida.
\item[(b)] Calcular la media y la varianza de la distancia de la potencia de fuego de la Estrella de la Muerte.
\item[(c)] Calcular la mediana de la distancia de la potencia de fuego de la Estrella de la Muerte.
\end{itemize}
\end{frame} 

\section{Soluciones} 
\begin{frame}
\frametitle{Soluci\'on Problema 1}
\end{frame}
 
\begin{frame}[plain]
\begin{center}
\Huge Que la fuerza te acompa\~ne.
\end{center}
\end{frame}  
                        
\end{document}  
